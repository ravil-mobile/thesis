\chapter{Solver selection and configuration}
\label{chapter:solver configuration}

% from numerical integration to system of linear equations
Numerical integration of time dependent partial differential equations can be achieved using different numerical schemes, namely: Runge-Kutta methods. In numerical analysis, the Runge–Kutta methods are a family of implicit and explicit iterative methods \cite{wiki:runge-kutta}.  \\


Implicit and explicit methods have their advantages and disadvantages which can be found in the corresponding literature.
% reference on something

 In general, implicit methods are robust in case of stiff equations, however, on another hand they are more expensive in terms of computational cost since they introduce a non-linear part that has to be solve using the Newton's method as an example. There is also a class of integration methods called semi-implicit that were designed to reduce the cost of implicit ones and be able to cope with stiff equations. But in all cases numerical integration boils down to solution of a couple systems of linear equations of a type:

\begin{equation} \label{eq:slq}
	Ax = b
\end{equation}

 where $A \in \mathbb{R^{N \times N}}$ is an invertible square matrix, $b \in \mathbb{R^{N}}$ represents the right-hand side, and $x \in \mathbb{R^{N}}$ is the solution vector. At this point we refer to a system of linear equations as just a system and we will use these two terms interchangeably. \\
 
In some cases, especially in case of implicit or semi-implicit methods, we have to compute several systems to perform one step of numerical integration. Thus, solution of a system \ref{eq:slq} is the computational core of any time integration scheme. In fact, it is the most expensive part and it is primary source of code optimization. \\ 

There are three major ways to solve a system of linear equations, namely: direct dense methods, sparse direct methods and iterative methods.\\

In the following sections \ref{subseq:direct methods}, \ref{subseq:iterative methods}, \ref{subseq:sparse methods} we are briefly going to review all three types of numerical solvers as well as their advantages and disadvantages and some aspects of their parallel implementations with a strong focus on their strong scaling behavior. Strong scaling is important for this research since we want to find a solver and its configuration to solve a certain system with a fixed size as fast as possible. Additionally to that, due to a large number of numerical integration steps, we want to find a robust solver to avoid any application crash during simulations.

At this point we can state requirements for a solver that we are looking for:

\begin{itemize}
	\item efficiency
	\item robustness
	\item numerical stability
	\item open source licenses
\end{itemize}

% TODO: observation of all possible methods to solve Ax = b. At this section you have to come to a conclusion that we have to use sparse direct methods
