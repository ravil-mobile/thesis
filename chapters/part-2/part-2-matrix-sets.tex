\section{Matrix sets} \label{subseq:matrix-sets}

During the stiudy, we are going to use two matrix sets: GRS and SuiteSparse. In our case, the SiteSparse matrix set is, in fact, few matrices downloaded from SuiteSparse Matrix Collection \cite{sparse-matrix-collection:1}, \cite{sparse-matrix-collection:2}. We tried to choose different matrices from the collection with respect to both the number of equations $n$ in a system and ratio $R$ between the number of non-zero elements $nnz$ and the number of equations.\\ 


\todo{describe what ATHLET is}
To generate GRS matrix set, we ran the most common GRS simulations in ATHLET and stopped the simulations somewhere in the middle saving corresponding shifted Jacobian matrices in the PETSc binary format.\\
\todo{describe what a shifted Jacobian matrix is}

Tables \ref{table:grs-matrix-set}  and \ref{table:suite-sparse-matrix-set} shows matrix properties of both matrix sets.\\


\begin{table}[ht]
\centering
\begin{tabular}{|l|l|l|l|}
\hline
Name     & n       & nnz      & nnz / n \\ \hline
pwr-3d   & 6009    & 117897   & 19.6200 \\ \hline
cube-5   & 9325    & 117897   & 6.0568  \\ \hline
cube-64  & 100657  & 1388993  & 13.7993 \\ \hline
cube-645 & 1000045 & 13906057 & 13.9054 \\ \hline
k3-2     & 130101  & 787997   & 6.0568  \\ \hline
k3-18    & 1155955 & 7204723  & 6.2327  \\ \hline
\end{tabular}
\caption{GRS matrix set}
\label{table:grs-matrix-set}
\end{table}



\begin{table}[ht]
\centering
\begin{tabular}{|l|l|l|l|l|}
\hline
Name        & n       & nnz      & nnz / n & Field                      \\ \hline
cant        & 62451   & 4007383  & 64.1684 & 2D/3D Problem              \\ \hline
consph      & 83334   & 6010480  & 72.1251 & 2D/3D Problem              \\ \hline
CurlCurl\_3 & 1219574 & 13544618 & 11.1060 & Model Reduction Problem    \\ \hline
Geo\_1438   & 1437960 & 63156690 & 43.9210 & 2D/3D Problem              \\ \hline
memchip     & 2707524 & 13343948 & 4.9285  & Circuit Simulation Problem \\ \hline
nlpkkt80    & 1062400 & 28192672 & 26.5368 & Optimization Problem       \\ \hline
PFlow\_742  & 742793  & 37138461 & 49.9984 & 2D/3D Problem              \\ \hline
pkustk10    & 80676   & 4308984  & 53.4110 & Structural Problem         \\ \hline
torso3      & 259156  & 4429042  & 7.0903  & 2D/3D Problem              \\ \hline
x104        & 108384  & 8713602  & 80.3956 & Structural Problem         \\ \hline
\end{tabular}
\caption{SuiteSparse matrix set}
\label{table:suite-sparse-matrix-set}
\end{table}


From now, we are going to introduce and use a definition of \textbf{skinny sparse matrices}. It is matrices with relatively low  ration $R$ i.e. less than 15. \\ 


The objective of this study is to find and configure a solver which can fulfill all requirements listed above for the GRS matrix set. From time to time, we will use the SuiteSparse set for comparison reasons. It seems to us that GRS matrix set is different because of some specifics of 1D pipeline discretization. For example, GRS matrices are skinny and blocked, where each block is a small, approximately 3-by-3, dense matrix. Additionally, some rows can contain only one element i.e. on the diagonal. It is a result of dynamic pipeline switching of a reactor cooling system. As we will see in section \ref{subseq:sparse methods}, sparse direct solvers can be quite sensitive to the sparse structure of a matrix.\\


We used different hardware to measure performance of different solvers. The first machine was the GRS cluster (HW1) which was our main target. We used a LRZ Linux cluster (HW2) every time when we got some ambiguous results in order to check whether a problem was hardware or software specific. Table \ref{table:hardware-spec} shows a single node specification of both machines. 


\todo{fill in the tables \ref{table:hardware-spec}}
\begin{table}[ht]
\centering
\begin{tabular}{|l|l|l|}
\hline
                    & HW1 (GRS) & HW2 (LRZ Linux) \\ \hline
Architecture        &           &                 \\ \hline
CPU(s)              &           &                 \\ \hline
On-line CPU(s) list &           &                 \\ \hline
Thread(s) per core  &           &                 \\ \hline
Core(s) per socket  &           &                 \\ \hline
Socket(s)           &           &                 \\ \hline
NUMA node(s)        &           &                 \\ \hline
Model               &           &                 \\ \hline
Model name          &           &                 \\ \hline
Stepping            &           &                 \\ \hline
CPU MHz             &           &                 \\ \hline
CPU max MHz         &           &                 \\ \hline
CPU min MHz         &           &                 \\ \hline
Virtualization      &           &                 \\ \hline
L1d cache           &           &                 \\ \hline
L1i cache           &           &                 \\ \hline
L2 cache            &           &                 \\ \hline
L3 cache            &           &                 \\ \hline
NUMA node0 CPU(s)   &           &                 \\ \hline
\end{tabular}
\caption{Hardware specification}
\label{table:hardware-spec}
\end{table}

\newpage