\label{subseq:hybrid-method-description}

Nowadays, iterative methods is a common choice for solving sparse systems of linear equations because of their possible fast convergence and high parallel efficiency. However, application of such method always demands preconditioning of ill-conditioned systems to make methods converge to numerical accurate solutions. It can be clearly observed from table \ref{table:grs-matrix-set} that numerical integration of thermo-hydraulic simulations in \gls{athlet} entails solving such ill-conditioned systems  based on estimated condition numbers of matrices form \gls{grs} matrix set.\\


As the first step of the study, we tested various preconditioning algorithms together with their tuning parameters, mentioned in table \ref{table:preconditioners}, applied to \gls{grs} matrix set. \gls{gmres} was chosen as an iterative solver with values of relative and absolute convergence tolerances in the residual norm to be equal to $1E-8$ and $1E-4$, respectively. A coarse grid search was used with maximum 3 values for each tuning parameter starting from the default towards more accurate values in order to refine settings of each preconditioning algorithm. Testing results showed that none of them could lead to convergence for the entire set of matrices.\\


One can assume that a finer grid search can result in finding a suitable preconditioning algorithm with settings that can lead to convergence of \gls{gmres} solver for the entire set. However, it is important to point out the matrices were generated by running the most common \gls{grs} thermo-hydraulic test-scenarios and saving them somewhere during the time integration process. Hence, there is no guarantee that the settings found in such a way can always lead to convergence of \gls{gmres} solver in all time steps of any thermo-hydraulic simulation. Therefore, iterative methods may not satisfy \textit{robustness} criteria stated in chapter \ref{chapter:problem-statment} as a non-functional requirement to the time integration solver used in \gls{athlet}.\\


Taking into account the above reasoning, we have come to the conclusion that sparse direct methods is the best choice for our problem, in spite of the limited tree-task parallelism described in subsection \ref{subseq:direct-parallel-aspects}, because the methods stably result in numerical accurate solutions even in case of ill-conditioned linear systems. Hence, the next objective of the study is to find a suitable sparse direct method and its implementation, and adapt it for \gls{hw1} compute-cluster environment in terms of efficient parallel execution. \\
