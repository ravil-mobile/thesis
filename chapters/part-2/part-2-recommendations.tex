\label{subseq:mm-recommendations}


According to results and conclusion obtained in this study, we are able to make a general guidance for ATHLET-NuT users and developers. \\


First of all, we insist to configure PETSc-MUMPS module of NuT with OpenBLAS library and use only flat-MPI mode together with \textit{spread} process pinning strategy because, as it was shown in sections \ref{subseq:mm-mumps-process-pinning} and \ref{subseq:blas-comparison}, these settings always results in better parallel performance regardless of the matrix size and structure.\\


Secondly, we recommend the user to check table \ref{table:mm-recommendations-size-table} before running a simulation in order to define a matrix type of a problem i.e. \textit{small}, \textit{medium} or \textit{large}. Based on the matrix type, the user can select both a fill reducing reordering algorithm and the number of MPI processes which, we believe, can be pretty close to the optimal MUMPS settings. \\


\begin{table}[h!]
\centering
\begin{tabular}{|c|c|}
\hline
Matrix type & \begin{tabular}[c]{@{}c@{}}Number of \\ equations\end{tabular}      \\ \hline
Small       & \begin{tabular}[c]{@{}c@{}}Less than\\ 25000\end{tabular}           \\ \hline
Medium      & \begin{tabular}[c]{@{}c@{}}form \\ 50000\\ to\\ 250000\end{tabular} \\ \hline
Large       & \begin{tabular}[c]{@{}c@{}}More\\ than\\ 500000\end{tabular}        \\ \hline
\end{tabular}
\caption{Matrix type reference table}
\label{table:mm-recommendations-size-table}
\end{table}


We recommend to use only PT-Scotch and no more than 4 MPI processes for \textbf{small} systems. The same reordering algorithm, PT-Scotch, is better to apply to \textbf{medium} sized systems of equations as well, however, the optimal number of MPI processes lays in a range from 5 to 10. In case of \textbf{large} systems, we insistently recommend to switch to ParMetis package to perform fill-in reduction and start parallel execution from 8 MPI processes.\\




