\chapter{Methodology and Experimental Setup}\label{subseq:matrix-sets-and-hardware}

%Static solver configuration
\acrshort{athlet} is a tool designed for computer-based simulations of transient thermo-hydraulic problems where topology of hydraulic circuits can be changed during a numerical simulation. As a result, the Jacobian matrix often changes with respect to both numerical values and the matrix sparsity structure between integration time steps. Hence, a configuration of a linear solver in run-time becomes a time consuming and compute-expensive problem since \acrshort{athlet} usually generates hundreds of matrices during a simulation. Moreover, results of such dynamic solver configurations may be difficult to analyze and interpret.\\


In this study, it was decided to stick to a so-called static solver configuration approach: configuring a solver with only a small set of matrices, i.e. GRS matrix set, randomly saved during simulations of the most common \acrshort{grs} test scenarios. In the general case, this approach may lead to not accurate conclusions, however, it is only one feasible.\\


Besides \acrshort{grs} matrix set, the second set was  used for verification purposes of testing results. The set, called SuiteSparse matrix set, was generated by donwloading a dozen of matrices from SuiteSparse Matrix Collection \cite{sparse-matrix-collection:1}, \cite{sparse-matrix-collection:2} where we tried to pick out different matrices with respect to both the number of equations $n$ and matrix density i.e. ratio between the number of non-zero elements $nnz$ and the number of equations in a system.\\ 


The main matrix properties as well as matrix sparsity patterns are shown in tables \ref{table:grs-matrix-set}, \ref{table:suite-sparse-matrix-set} and appendix \ref{app:sparsity-patterns}.\\


\begin{table}[ht]
\small
\centering
\begin{tabular}{|c|c|c|c|c|c|}
\hline
Name     & \textit{n}       & \textit{nnz}      & \textit{nnz} / \textit{n} & \begin{tabular}[c]{@{}c@{}}Approximate\\ Condition\\  Number\end{tabular} & Structure     \\ \hline
pwr-3d   & 6009    & 32537    & 5.4147  & 1.019e+07                                                                 & SYMM-PTRN \\ \hline
cube-5   & 9325    & 117897   & 12.6431 & 1.592e+09                                                                 & SYMM-PTRN \\ \hline
cube-64  & 100657  & 1388993  & 13.7993 & 7.406e+08                                                                 & SYMM-PTRN \\ \hline
cube-645 & 1000045 & 13906057 & 13.9054 & 6.474e+08                                                                 & SYMM-PTRN \\ \hline
k3-2     & 130101  & 787997   & 6.0568  & 1.965e+15                                                                 & SYMM-PTRN \\ \hline
k3-18    & 1155955 & 7204723  & 6.2327  & 1.947e+12                                                                 & SYMM-PTRN \\ \hline
\end{tabular}
\caption{\acrshort{grs} matrix set \textit{(where SYMM - symmetric; NON-SYMM - non-symmetric; SYMM-PTRN- non-symmetric but with symmetric sparsity pattern)}}
\label{table:grs-matrix-set}
\end{table}




\begin{table}[ht]
\centering
\small
\begin{tabular}{|c|c|c|c|c|c|c|}
\hline
Name        & \textit{n}       & \textit{nnz}      & \textit{nnz} / \textit{n} & \begin{tabular}[c]{@{}c@{}}Approximate\\ Condition\\ Number\end{tabular} & Structure & Problem                                                      \\ \hline
cant        & 62451   & 4007383  & 64.1684 & 5.082e+05 & SYMM      & -                                                            \\ \hline
consph      & 83334   & 6010480  & 72.1251 & 2.438e+05 & SYMM      & -                                                            \\ \hline
CurlCurl\_3 & 1219574 & 13544618 & 11.1060 & 2.105e+05                                                                & SYMM      & \begin{tabular}[c]{@{}c@{}}Model\\ Reduction\end{tabular}    \\ \hline
Geo\_1438   & 1437960 & 63156690 & 43.9210 & 4.677e+05                                                                 & SYMM      & -                                                            \\ \hline
memchip     & 2707524 & 13343948 & 4.9285  & 1.305e+07                                                                & NON\_SYMM & \begin{tabular}[c]{@{}c@{}}Circuit\\ Simulation\end{tabular} \\ \hline
PFlow\_742  & 742793  & 37138461 & 49.9984 & 5.553e+06                                                                & SYMM      & -                                                            \\ \hline
pkustk10    & 80676   & 4308984  & 53.4110 & 5.589e+02 & SYMM      & Structural                                                   \\ \hline
torso3      & 259156  & 4429042  & 7.0903  & 2.456e+03                                                                      & NON\_SYMM & -                                                            \\ \hline
x104        & 108384  & 8713602  & 80.3956 & 3.124e+05 & SYMM      & Structural                                                   \\ \hline
\end{tabular}
\caption{SuiteSparse matrix set \textit{(where SYMM - symmetric; NON-SYMM - non-symmetric; SYMM-PTRN- non-symmetric but with symmetric sparsity pattern)}}
\label{table:suite-sparse-matrix-set}
\end{table}

Approximations of condition numbers, shown in tables \ref{table:grs-matrix-set} and \ref{table:suite-sparse-matrix-set},  were computed using Rayleigh–Ritz procedure \cite{rayleigh-ritz-procedure}. \acrshort{gmres} solver configured with $1000$ iteration steps before the restart was applied to un-preconditioned systems to generate a Krylov subspace for each matrix. Then, the resulting Hessenberg matrices were used for approximating eigenspaces and the corresponding eigenvalues. The approximations should be treated as lower bounds since the algorithm overestimates the smallest eigenvalues.\\


%The objective of this study is to find and configure a sparse linear solver which can fulfill all requirements listed above for the \acrshort{grs} matrix set. It is worth pointing out, as it was mentioned in section \ref{sec:athlet-overview}, \acrshort{athlet} performs many mathematical transformations upon the original system and, finally, generates an approximation of a Jacobian matrix. For that reason, one can assume that \acrshort{grs} matrix set can be structurally different from matrices coming naturally from finite-volume, finite-elements discretization or optimization problems. Therefore, SuiteSparse matrix set was used, from time to time, to examine this statement.\\


Two different hardware were available for this study. The first machine was a compute-cluster installed in \acrshort{grs} (\gls{hw1}) which was the main target. Additionally, \acrshort{lrz} CoolMUC-2 Linux cluster (\gls{hw2}) was used every time when some ambiguous results were obtained in order to check whether a problem was hardware, software or algorithmic specific. Table \ref{table:hardware-spec} shows compute-node specifications of both compute-clusters.\\


\begin{table}
\centering
\small
\begin{tabular}{|l|c|c|}
\hline
                    & HW1 (GRS) & HW2 (LRZ Linux) \\ \hline
Architecture        & x86\_64 & x86\_64 \\ \hline
CPU(s)              & 20 &  28 \\ \hline
On-line CPU(s) list & 0-19 &  0-27 \\ \hline
Thread(s) per core  & 1 &  1 \\  \hline
Core(s) per socket  & 10 & 14 \\ \hline
Socket(s)           & 2 &  2 \\ \hline
NUMA node(s)        & 2 &  4 \\ \hline
Model               & 62 &  63 \\ \hline
Model name          & E5-2680 v2 & 
E5-2697 v3 \\ \hline
Stepping            & 4 &  2 \\ \hline
CPU MHz             & 1200.0 &  2036.707 \\ \hline
Virtualization      & VT-x &  VT-x \\ \hline
L1d cache           & 32K &  32K \\ \hline
L1i cache           & 32K &  32K \\ \hline
L2 cache            & 256K &  256K \\ \hline
L3 cache            & 25600K &  17920K \\ \hline
NUMA node0 CPU(s)   & 0-9 &  0-6 \\ \hline
NUMA node1 CPU(s)   & 10-19 &  7-13 \\ \hline
NUMA node2 CPU(s)   & - &  14-20 \\ \hline
NUMA node3 CPU(s)   & - &  21-27 \\ \hline
RAM per node, GB   & 128 &  64 \\ \hline
\end{tabular}
\caption{Hardware specification}
\label{table:hardware-spec}
\end{table}


For this study, OpenMPI implementation of the \acrshort{mpi} standard was used because of its open-source license and comprehensive documentation. The library has many options for processes pinning which was intensively used during the study.\\


To make process pinning explicit and deterministic, a python script was developed to generate rank-files automatically based on the number of \acrshort{mpi} processes, \acrshort{openmp} threads per \acrshort{mpi} process, the maximum number of processing elements and the number of \acrshort{numa} domains. The scrip always leaves appropriate gaps between \acrshort{mpi} processes to allow each process to fork the corresponding number of threads within a parallel region.\\


\begin{figure}[h!]
\centering
	\begin{tabular}{cc}
			\subfloat[\textit{Spread} mode]{\includegraphics[width=0.45\textwidth]{figures/chapter-2/spread-mode.png}} &
		\subfloat[\textit{Close} mode]{\includegraphics[width=0.45\textwidth]{figures/chapter-2/close-mode.png}} \\
	\end{tabular}
	\caption{An example of pinning 5 \acrshort{mpi} processes with 2 \acrshort{openmp} threads per process in case of \gls{hw1} hardware}
	\label{fig:python-script-rankfile-example}
\end{figure}


A rank-file specifies explicit mapping between \acrshort{mpi} processes, ranks, and actual processing elements, cores, of a machine. The script has two modes, namely: \textit{spread} and \textit{close}. Given a certain number of ranks, \textit{spread} mode tries to distribute them as spread as possible across multiple available \acrshort{numa} domains in a round-robin fashion. In contrast to \textit{spread} strategy, \textit{close} one groups ranks as close as possible to keep the maximum number of ranks within a single \acrshort{numa} domain. Figure \ref{fig:python-script-rankfile-example} shows an example of mapping 5 \acrshort{mpi} ranks and 2 \acrshort{openmp} threads per rank onto a compute node equipped with 20 cores and 2 \acrshort{numa} domains (\gls{hw1}).\\


In this study, \acrshort{petsc} 3.10 and OpenMPI 3.1.1 libraries were chosen and compiled with Intel 18.2 compiler.\\
