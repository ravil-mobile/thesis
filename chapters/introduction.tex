\chapter{Introduction}\label{chapter:introduction}

% what is nuclear energy and why it is attractive?
Nowadays, nuclear energy is one of the main sources of electricity. It comes from splitting atoms in a reactor which, as a result, heats water up to the point where it is converted into pressurized steam. In its turn, the steam rotates turbines which, finally, produce electricity. According to the recent estimations, thermal efficiency of modern nuclear power plants lies in the range of 35-45\% which is comparable to conventional fossil fueled power plants \cite{intro:efficiency-of-nuclear-power-plants}. In spite of considerable initial investments,  nuclear power plants have low operating costs and long service life which make them particularly cost effective.\\


% Advantages
In recent years, nuclear power plants have become an attractive means of power generation because of relatively low emission of carbon dioxide. As a result, a level of green house gase emissions to the atmosphere and thus the contribution of nuclear power plants to the global warming are relatively small \cite{intro:pros-and-cons-of-nuclear-power}.\\


% Example of nuclear energy utilization in the EU
Today, nuclear power plants generate almost 30\% of the electricity produced in the European Union (EU). There are almost 130 nuclear reactors in operation in 14 EU countries, namely: Belgium, Bulgaria, Czech Republic, Finland, France, Germany, Hungary, Netherlands, Romania, Slovakia, Slovenia, Spain, Sweden, and the United Kingdom \cite{intro:eu-nuclear-industry-general}.\\


% danger of nuclear power
% Disadvantages: Accidents Can Happen
% idea: requirements to perform hundreds of experiments and foresee all possible outcomes.
The main problem assosiated with nuclear power is radioactive waste which is extremely dangerous for people and the environment and has to be carefully looked after for several thousand years after utilization. Any accident in a plant can lead to grave consequences at a scale similar to the Chernobyl disaster. For this reason, nuclear safety is one of the most important topics in this area. It demands a huge amount of testing and analysis to be performed before and during an operation of a nuclear power plant in order to predict any possiblity of unwanted outcomes and devise preventive measures against such accidents. The topic has become even more prominent since 2011 Fukushima accident. In response to the disaster, numerous stress tests were conducted to measure the ability of the EU nuclear industry to withstand any kind of natural disaster \cite{intro:eu-nuclear-industry-general}.\\



% 1. \acrshort{grs}: general info and the goal
% 2. Simulation tools
Since 1977, Gesellschaft für Anlagen- und Reaktorsicherheit (\acrshort{grs}) has been the main German scientific research institute in the field of nuclear safety and radioactive waste management \cite{grs:grs-general-info}. Today, the organization carries out advanced research and analysis in the field of reactor safety, radioactive waste management as well as radiation and environmental protection \cite{grs:grs-general-info}. Due to the inability to create various nuclear accident test scenarios, which by their very nature could be catastrophic, \acrshort{grs} develops and provides numerous simulation software products to cope with this problem. A short description of the main software packages developed by \acrshort{grs} is provided in Table \ref{table:introduction-grs-software}.\\

%Due to a huge amount of testing, various types of possible accidents and disability to conduct natural experiments, \acrshort{grs} provides and develops numerous simulation software products to cope with this problem. Table \ref{table:introduction-grs-software} represents the main software packages that have been developed by \acrshort{grs} and their short description.\\

\begin{table}[htb]
\centering
\begin{tabular}{|c|l|}
\hline
Name          & \multicolumn{1}{c|}{Description}                                                                                                                                                             \\ \hline
ATHLET        & \begin{tabular}[c]{@{}l@{}}Thermohydraulic safety analyses for the primary\\ circuit of  LWRs\end{tabular}                                                                                   \\ \hline
ATHLET-CD     & \begin{tabular}[c]{@{}l@{}}Analyses of accidents with core meltdown \\ and fission product release for LWRs\end{tabular}                                                                     \\ \hline
ATLAS         & \begin{tabular}[c]{@{}l@{}}Analysis simulator for interactive handling \\ and  visualisation of several computer codes\end{tabular}                                                          \\ \hline
COCOSYS       & \begin{tabular}[c]{@{}l@{}}Analyses of severe incidents in the \\ containment of LWRs\end{tabular}                                                                                           \\ \hline
DORT/TORT     & \begin{tabular}[c]{@{}l@{}}Solution of time-dependant neutron transport\\ equations for 2D/3D transients analyses\end{tabular}                                                               \\ \hline
QUABOX/CUBBOX & 3-D neutron kinetics core model                                                                                                                                                              \\ \hline
SUSA          & Uncertainty and sensitivity analyses                                                                                                                                                         \\ \hline
TESPA-ROD     & Core rod code for design basis accidents                                                                                                                                                     \\ \hline
NUT     & Container of various numerical tools and algorithms \\ \hline
\end{tabular}
\caption[A list of software developed by \acrshort{grs}]{A list of software developed by \acrshort{grs}, \cite{grs:grs-software-products}, where LWR stands for a Light Water Reactor}
\label{table:introduction-grs-software}
\end{table}



% objective of the study
The main focus of this thesis is dedicated to \acrshort{athlet} and \acrshort{nut} software packages. The goal of the study is to identify the most compute-intensive parts of the \acrshort{athlet}-\acrshort{nut} code and possibly accelerate its execution time.\\


The next chapter continues the introduction and gives a general overview of \acrshort{athlet}-\acrshort{nut} purpose, design, architecture and coupling. The introduction ends with a clear exposition of the problem statement presented in Chapter \ref{chapter:problem-statment} where the order of the remaining thesis is described in detail.\\

% whereas the subsequent one provides the problem statement and organization of the thesis.\\
