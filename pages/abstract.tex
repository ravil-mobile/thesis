\chapter{\abstractname}

\textit{\textbf{Key words.} numerical time integration, direct sparse linear methods, multi-frontal methods, \gls{mumps}, \gls{blas}, parallel performance, distributed-memory computations,  multi-threading, \gls{mpi}, \gls{openmp}, non-blocking communication}

\vspace{10mm}


An application of linearly implicit methods for time integration of stiff systems of \gls{ode}s  results in solving sparse systems of linear equations. The optimal selection and configuration of a parallel linear solver can sufficiently accelerate the time integration process and, therefore, reduce execution time of such simulations.\\



A comparison of iterative and direct sparse solvers has shown that direct ones are the most suitable because of their natural stability with respect to ill-conditioned linear systems that can occur during numerical time integration. Testing of different direct sparse solvers applied to systems generated by \gls{athlet} software has revealed that \gls{mumps}, an implementation of a multi-frontal method, performs better than the others in terms of the overall parallel execution time.\\


In this study, we have mainly focused on configurations of \gls{mumps} with the aim of improving parallel performance of the solver for thermo-hydraulic computations within a single node of \gls{grs} compute-cluster. We have investigated influences of  different fill reducing reordering algorithms, \gls{mpi} process pining, configurations of \gls{mumps} with different \gls{blas} library implementations and \gls{mumps}/\gls{openmp} hybrid computing strategies on parallel performance of the solver.\\


Additionally, we have shown that an intelligent application of non-blocking \gls{mpi} communications in some parts of the existing thermo-hydraulic simulation code, \gls{athlet}, can additionally solve issues of inefficient data transfer preserving the current software design and implementation without drastic changes in the source code.\\