\chapter{\abstractname}

An application of linearly implicit methods for time integration of stiff systems of ODEs  results in solving sparse systems of linear equations. The optimal selection and configuration of a parallel linear solver can sufficiently accelerate the time integration process and, therefore, reduce execution time of such simulations.\\



A comparison of iterative and direct sparse solvers has shown that direct ones are the most suitable because of their natural stability with respect to ill-conditioned linear systems that can occur during numerical time integration. Testing of different direct sparse solvers applied to systems generated by ATHLET software has revealed that MUMPS, an implementation of a multi-frontal method, performs better than the others in terms of the overall parallel execution time.\\


In this study, we have mainly focused on configurations of MUMPS with the aim of improving parallel performance of the solver for thermo-hydraulic computations within a single node of GRS compute-cluster. We have investigated influences of  different fill reducing reordering algorithms, MPI process pining, configurations of MUMPS with different BLAS library implementations and MPI/OpenMP hybrid computing strategies on parallel performance of the solver.\\


Additionally, we have shown that an intelligent application of non-blocking MPI communications in some parts of the existing thermo-hydraulic simulation code, ATHLET, can additionally solve issues of inefficient data transfer preserving the current software design and implementation without drastic changes in the source code.\\

\vspace{20mm}

\textit{\textbf{Key words.} direct sparse linear methods, multi-frontal methods, parallel performance, distributed-memory computations,  multi-threading, MPI, OpenMP, BLAS, non-blocking communication}