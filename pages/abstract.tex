\chapter{\abstractname}

\textit{\textbf{Key words.} numerical time integration, direct sparse linear methods, multifrontal methods, \acrshort{mumps}, \acrshort{blas}, parallel performance, distributed-memory computations,  multi-threading, \acrshort{mpi}, \acrshort{openmp}, non-blocking communication}

\vspace{10mm}



An application of linearly implicit methods for time integration of stiff systems of \acrshort{ode}s  results in solving sparse systems of linear equations. An optimal selection and configuration of a parallel linear solver can considerably accelerate the time integration process. A comparison of iterative and direct sparse linear solvers has shown that direct ones are the most suitable for this purpose because of their natural robustness to ill-conditioned linear systems that can occur during numerical time integration. Testing of different direct sparse solvers applied to systems generated by \acrshort{athlet} software has revealed that \acrshort{mumps}, an implementation of the multifrontal method, performs better than the others in terms of the overall parallel execution time.\\


%In this study, we have mainly focused on configurations of \acrshort{mumps} with the aim of improving parallel performance of the solver for thermo-hydraulic computations within a single node of \acrshort{grs} compute-cluster. We have investigated influences of  different fill reducing reordering algorithms, \acrshort{mpi} process pining, configurations of \acrshort{mumps} with different \acrshort{blas} library implementations and \acrshort{mpi}/\acrshort{openmp} hybrid computing on parallel performance of the solver.\\

In this study, we have mainly focused on configuring \acrshort{mumps} with the aim of improving parallel performance of the solver for thermo-hydraulic computations within a single node of \acrshort{grs} compute-cluster. However, the overall approach,  proposed in the study, may be considered as a general framework for a selection and adaptation of a linear sparse solver for solving problem-specific systems of linear equations on distributed-memory machines.\\


Additionally, we have shown that an intelligent application of non-blocking \acrshort{mpi} communication in some parts of the existing thermo-hydraulic simulation code, \acrshort{athlet}, can additionally solve issues of inefficient data transfer preserving the current software design and implementation without drastic changes of the source code.\\
